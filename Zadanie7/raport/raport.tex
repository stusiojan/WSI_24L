\documentclass{article}
\usepackage[T1]{fontenc}
\usepackage[utf8]{inputenc}
\usepackage[margin=2cm]{geometry}
\usepackage{graphicx}
\usepackage{parskip}
\usepackage{booktabs}
\usepackage{amsmath}

\title{Zadanie 7 - Raport}
\author{Jan Stusio}
\date{Czerwiec 2024}

\begin{document}

\maketitle

\section{Wstęp}

Celem zadania jest implementacja klasyfikatora Gaussowskiego Naiwnego Bayesa oraz wnioskowanie dla tego klasyfikatora oraz uczenie parametrów. 
Klasyfikator ten zakłada brak zależności między zmiennymi objaśniającymi oraz że wartości atrybutów pochodzą z rozkładu normalnego. 
Zadanie polegało również na porównaniu wyników tego klasyfikatora z klasyfikatorem drzewa decyzyjnego oraz SVM z zadania 4 dla najlepszych parametrów.

\section{Implementacja}

Klasa \texttt{GaussianNB} posiada metody \texttt{fit} oraz \texttt{predict}. 
Metoda \texttt{fit} uczy parametry klasyfikatora, natomiast metoda \texttt{predict} dokonuje klasyfikacji. 

Metody są wywoływane przez \texttt{cross\_val\_score} z biblioteki \texttt{sklearn}, która przeprowadza walidację krzyżową z 5 podziałami, aby ocenić wydajność modelu.

\section{Wyniki}

\begin{table}[h!]
    \centering
    \begin{tabular}{lrrrrrrrr}
        \toprule
        Model & Accuracy & Accuracy ± & Precision & Precision ± & Recall & Recall ± & F1 & F1 ± \\
        \midrule
        Gaussian Naive Bayes & 0.953333 & 0.026667 & 0.958384 & 0.023983 & 0.953333 & 0.026667 & 0.953047 & 0.026862 \\
        Decision Tree & 0.953333 & 0.033993 & 0.968350 & 0.035671 & 0.960000 & 0.032660 & 0.966583 & 0.036606 \\
        SVM & 0.980000 & 0.016330 & 0.981818 & 0.014845 & 0.980000 & 0.016330 & 0.979950 & 0.016371 \\
        \bottomrule
    \end{tabular}
    \caption{Porównanie wyników klasyfikatorów na zbiorze danych Iris}
\end{table}


\section{Wnioski}

Na podstawie przeprowadzonej analizy oraz uzyskanych wyników można sformułować następujące wnioski:

\begin{itemize}
    \item Klasyfikator Gaussowskiego Naiwnego Bayesa osiągnął wysoką dokładność klasyfikacji (95.33\%) na zbiorze danych Iris. 
    Jest to wynik porównywalny z klasyfikatorem drzewa decyzyjnego (95.33\%), jednak nieco gorszy niż klasyfikator SVM (98.00\%).
    
    \item Wartości odchylenia standardowego dla wszystkich miar jakości klasyfikacji (Accuracy, Precision, Recall, F1) są niewielkie (wszystkie poniżej 0,04),
    co wskazuje na stabilność i powtarzalność wyników uzyskiwanych przez klasyfikator.
    
    \item Gaussowski Naiwny Bayes, pomimo swoich uproszczonych założeń dotyczących niezależności zmiennych objaśniających, 
    okazał się być porównywalny do bardziej zaawansowanych metod klasyfikacji (badanych w zadaniu 4).
    
    \item Implementacja Gaussowskiego Naiwnego Bayesa nie wymaga strojenia hiperparametrów, więc może być szybszy do implementacjiś od SVM i drzewa decyzyjnego.
\end{itemize}

Gaussowski Naiwny Bayes jest klasyfikatorem niewiele gorszym od SVM i drzewa decyzyjnego, więc dostarcza w miarę dokładne wyniki klasyfikacji. 

\end{document}
