\documentclass{article}
\usepackage[T1]{fontenc}
\usepackage[margin=2cm]{geometry}
\usepackage{graphicx}
\usepackage{parskip} % removes paragraph indentation

\title{Zadanie 2 - Raport}
\author{Jan Stusio}
\date{Marzec 2024}

\begin{document}

\maketitle

\section{Wstęp}

Celem zadania jest zaimplementowanie algorytmu ewolucyjnego, który jest bezgradientową metodą optymalizacji.


Analizowane tą metodą zostaną funkcje:

1. Rastrigina(https://www.sfu.ca/~ssurjano/rastr.html),

dla zakresu $x \in [-5,12;5,12]^2, x \in R^2$

$$
f(x) = 10d + \sum_{i=1}^{d} \left( x_i^2 - 10 \cos(2 \pi x_i) \right)
$$

2. Griewanka(https://www.sfu.ca/~ssurjano/griewank.html),

dla zakresu $x \in [-50, 50]^2, x \in R^2$

$$
f(x) =\sum_{i=1}^{d} \frac{x_i^2}{4000} - \prod_{i=1}^{d} \cos\left(\frac{x_i}{\sqrt{i}}\right) +1
$$

$3^*$. Drop-Wave(https://www.sfu.ca/~ssurjano/drop.html),

dla zakresu $x \in [-5.12, 5.12]^2, x \in R^2$

$$
g(x) = -\frac{1 + \cos\left(12 \sqrt{x_1^2 + x_2^2}\right)}{0.5(x_1^2 + x_2^2) + 2}
$$

\pagebreak

\section{Implementacja}


Rozważam tylko wymiar $d = 2$, zatem analizowane funkcje można uprościć:

Rastrigin

$$
f(x) = 20 + x_1^2 - 10 \cos(2 \pi x_1) + x_2^2 - 10 \cos(2 \pi x_2)
$$

Griewank

$$
f(x) =\frac{1}{4000} (x_1^2 + x_2^2) - \cos\left(x_1\right) \cos\left(\frac{x_2}{\sqrt{2}}\right) + 1
$$

\section{Badane parametry}

\section{Testy}

\section{Wizualizacje parametrów}

\section{Wnioski}

\end{document}
